\documentclass[a4paper,12pt]{report}

\usepackage{cmap}
\usepackage[T2A]{fontenc}
\usepackage[utf8]{inputenc}
\usepackage[english,russian]{babel}
\usepackage{listings}
\usepackage{amsmath}
\usepackage{float}
\usepackage{csquotes}
\usepackage{mathtools}
\usepackage{hyphenat}
\usepackage{amsfonts}
\usepackage{upgreek}

\usepackage{xcolor}
\usepackage{hyperref}

\usepackage{graphicx}
\graphicspath{ {./images/} }

\definecolor{dkgreen}{rgb}{0,0.6,0}
\definecolor{gray}{rgb}{0.5,0.5,0.5}
\definecolor{mauve}{rgb}{0.58,0,0.82}

\lstset{
    language=Python,                 % выбор языка для подсветки (здесь это С)
    basicstyle=\small\sffamily, % размер и начертание шрифта для подсветки кода
    numbers=left,               % где поставить нумерацию строк (слева\справа)
    numberstyle=\tiny,           % размер шрифта для номеров строк
    stepnumber=1,                   % размер шага между двумя номерами строк
    numbersep=5pt,                % как далеко отстоят номера строк от подсвечиваемого кода
    aboveskip=3mm,
    belowskip=3mm,
    showstringspaces=false,
    columns=flexible,
    captionpos=b, 
    basicstyle={\small\ttfamily},
    numbers=left,
    numberstyle=\tiny\color{gray},
    keywordstyle=\color{blue},
    commentstyle=\color{mauve},
    stringstyle=\color{dkgreen},
    breaklines=true,
    breakatwhitespace=true,
    tabsize=3
}

\title{Лабораторная работа №11\\Модуляция и выборка (квантование)}
\author{Кобыжев Александр}
\date{\today}

\begin{document}

\maketitle
\tableofcontents
\listoffigures
\lstlistoflistings

\maketitle

\chapter{Упражнение 11.1}

В данном упражнении нас просят открыть \texttt{chap11.ipynb}, прочитать пояснения, а также запустить примеры.

\chapter{Упражнение 11.2}

В данном упражнении нас просят посмотреть видео под названием \texttt{D/A and A/D | Digital Show and Tell}, которое доступно по \href{https://www.youtube.com/watch?v=cIQ9IXSUzuM}{\texttt{ссылке}}.

Это видео о споре качества цифрового и аналогового звука. В нём на примерах объясняется, почему аналоговый звук в допустимых пределах человеческого слуха (от 20 Гц до 20 кГц) может воспроизводиться с идеальной точностью с использованием 16-битного цифрового сигнала 44,1 кГц.

\chapter{Упражнение 11.3}

Для начала загрузим звук.

\begin{lstlisting}[caption=Загрузка звука]
wave = thinkdsp.read_wave('263868__kevcio__amen-break-a-160-bpm.wav')
wave.normalize()
wave.plot()
\end{lstlisting}

\begin{figure}[H]
        \centering
        \includegraphics[width=0.75\textwidth]{lab11_fig3_1.png}
        \caption{Визуализация звука}
        \label{fig:lab11_fig3_1}
\end{figure}

Этот сигнал дискредитируется с частотой 44100 Гц. Послушаем его.

\begin{lstlisting}[caption=Прослушивание звука]
wave.make_audio()
\end{lstlisting}

Составим спектр:

\begin{lstlisting}[caption=Спектр звука]
spectrum = wave.make_spectrum(full=True)
spectrum.plot()
\end{lstlisting}

\begin{figure}[H]
        \centering
        \includegraphics[width=0.75\textwidth]{lab11_fig3_2.png}
        \caption{Спектр звука}
        \label{fig:lab11_fig3_2}
\end{figure}

Уменьшим частоту дискретизации в 3 раза:

\begin{lstlisting}[caption=Уменьшение частоты дискретизации]
factor = 3
framerate = wave.framerate / factor
cutoff = framerate / 2 - 1
\end{lstlisting}

Перед дискретизацией мы применяем фильтр сглаживания, чтобы удалить частоты выше новой частоты свертки, которая равна частоте кадров / 2:

\begin{lstlisting}[caption=Отфильтрованный звук]
spectrum.low_pass(cutoff)
spectrum.plot()
\end{lstlisting}

\begin{figure}[H]
        \centering
        \includegraphics[width=0.75\textwidth]{lab11_fig3_3.png}
        \caption{Отфильтрованный звук}
        \label{fig:lab11_fig3_3}
\end{figure}

Прослушаем звук после фильтрации.

\begin{lstlisting}[caption=Прослушивание звука]
filtered = spectrum.make_wave()
filtered.make_audio()
\end{lstlisting}

Следующая функция имитирует процесс выборки:

\begin{lstlisting}[caption=Функция \texttt{sample}]
def sample(wave, factor):
    """Simulates sampling of a wave.
    
    wave: Wave object
    factor: ratio of the new framerate to the original
    """
    ys = np.zeros(len(wave))
    ys[::factor] = wave.ys[::factor]
    return thinkdsp.Wave(ys, framerate=wave.framerate)
\end{lstlisting}

Результат содержит копии спектра около 20 кГц:

\begin{lstlisting}[caption=Прослушивание звука]
sampled = sample(filtered, factor)
sampled.make_audio()
\end{lstlisting}

Они очень заметны и их можно рассмотреть, когда построим спектр:

\begin{lstlisting}[caption=Спектр звука]
sampled_spectrum = sampled.make_spectrum(full=True)
sampled_spectrum.plot()
\end{lstlisting}

\begin{figure}[H]
        \centering
        \includegraphics[width=0.75\textwidth]{lab11_fig3_4.png}
        \caption{Спектр звука}
        \label{fig:lab11_fig3_4}
\end{figure}

Мы можем избавиться от спектральных копий, снова применив фильтр сглаживания:

\begin{lstlisting}[caption=Применение фильтра сглаживания]
sampled_spectrum.low_pass(cutoff)
sampled_spectrum.plot()
\end{lstlisting}

\begin{figure}[H]
        \centering
        \includegraphics[width=0.75\textwidth]{lab11_fig3_5.png}
        \caption{Применение фильтра сглаживания}
        \label{fig:lab11_fig3_5}
\end{figure}

Мы только что потеряли половину энергии в спектре, но мы можем масштабировать результат, чтобы вернуть его:

\begin{lstlisting}[caption=Масштабирование результата]
sampled_spectrum.scale(factor)
spectrum.plot()
sampled_spectrum.plot()
\end{lstlisting}

\begin{figure}[H]
        \centering
        \includegraphics[width=0.75\textwidth]{lab11_fig3_6.png}
        \caption{Масштабирование результата}
        \label{fig:lab11_fig3_6}
\end{figure}

Теперь разница между спектром до и после дискретизации должна быть небольшой.

\begin{lstlisting}[caption=Разница между спектром до и после дискретизации]
spectrum.max_diff(sampled_spectrum)
\end{lstlisting}

Разница составила \texttt{1.8189894035458565e-12}. И разница между интерполированной волной и фильтрованной волной также должна быть небольшой.

\begin{lstlisting}[caption=Разница между интерполированной волной и фильтрованной волной]
filtered.plot()
interpolated.plot()
\end{lstlisting}

\begin{figure}[H]
        \centering
        \includegraphics[width=0.75\textwidth]{lab11_fig3_7.png}
        \caption{Разница между интерполированной волной и фильтрованной волной}
        \label{fig:lab11_fig3_7}
\end{figure}

\begin{lstlisting}[caption=Разница между интерполированной волной и фильтрованной волной]
filtered.max_diff(interpolated)
\end{lstlisting}

Разница составила \texttt{5.56290642113787e-16}.

Умножение на импульсы даёт 4 сдвинутых копии исходного спектра. Один из них проходит от отрицательного конца спектра к положительному, поэтому в спектре от выбранной волны есть 5 пиков.

\chapter{Выводы}

Во время выполнения лабораторной работы получены навыки работы с эффектом выборки и представил теорему выборки, которая объясняет сглаживание и частоту сворачивания. Также научился применять эти знания на практике.

\end{document}